\documentclass[11pt,addpoints,answers]{exam}
\usepackage[margin=1in]{geometry}
\usepackage{amsmath, amsfonts}
\usepackage{enumerate}
\usepackage{graphicx}
\usepackage{titling}
\usepackage{url}
\usepackage{xfrac}
\usepackage{geometry}
\usepackage{graphicx}
\usepackage{natbib}
\usepackage{amsmath}
\usepackage{amssymb}
\usepackage{amsthm}
\usepackage{paralist}
\usepackage{epstopdf}
\usepackage{tabularx}
\usepackage{longtable}
\usepackage{multirow}
\usepackage{multicol}
\usepackage[colorlinks=true,urlcolor=blue]{hyperref}
\usepackage{fancyvrb}
\usepackage{algorithm}
\usepackage{algorithmic}
\usepackage{float}
\usepackage{paralist}
\usepackage[svgname]{xcolor}
\usepackage{enumerate}
\usepackage{array}
\usepackage{times}
\usepackage{url}
\usepackage{comment}
\usepackage{environ}
\usepackage{times}
\usepackage{textcomp}
\usepackage{caption}
\usepackage[colorlinks=true,urlcolor=blue]{hyperref}
\usepackage{listings}
\usepackage{parskip} % For NIPS style paragraphs.
\usepackage[compact]{titlesec} % Less whitespace around titles
\usepackage[inline]{enumitem} % For inline enumerate* and itemize*
\usepackage{datetime}
\usepackage{comment}
% \usepackage{minted}
\usepackage{lastpage}
\usepackage{color}
\usepackage{xcolor}
\usepackage{listings}
\usepackage{tikz}
\usetikzlibrary{shapes,decorations,bayesnet}
%\usepackage{framed}
\usepackage{booktabs}
\usepackage{cprotect}
\usepackage{xcolor}
\usepackage{verbatimbox}
\usepackage[many]{tcolorbox}
\usepackage{cancel}
\usepackage{wasysym}
\usepackage{mdframed}
\usepackage{subcaption}
\usetikzlibrary{shapes.geometric}
\hypersetup{colorlinks=false, hidelinks=true}

%%%%%%%%%%%%%%%%%%%%%%%%%%%%%%%%%%%%%%%%%%%
% Formatting for \CorrectChoice of "exam" %
%%%%%%%%%%%%%%%%%%%%%%%%%%%%%%%%%%%%%%%%%%%

\CorrectChoiceEmphasis{}
\checkedchar{\blackcircle}

%%%%%%%%%%%%%%%%%%%%%%%%%%%%%%%%%%%%%%%%%%%
% Better numbering                        %
%%%%%%%%%%%%%%%%%%%%%%%%%%%%%%%%%%%%%%%%%%%

\numberwithin{equation}{section} % Number equations within sections (i.e. 1.1, 1.2, 2.1, 2.2 instead of 1, 2, 3, 4)
\numberwithin{figure}{section} % Number figures within sections (i.e. 1.1, 1.2, 2.1, 2.2 instead of 1, 2, 3, 4)
\numberwithin{table}{section} % Number tables within sections (i.e. 1.1, 1.2, 2.1, 2.2 instead of 1, 2, 3, 4)


%%%%%%%%%%%%%%%%%%%%%%%%%%%%%%%%%%%%%%%%%%%
% Common Math Commands                    %
%%%%%%%%%%%%%%%%%%%%%%%%%%%%%%%%%%%%%%%%%%%

%%%%%%%%%%%%%%%%%%%%%%%%%%%%%%%%%%%%%%%%%%
% Custom commands                        %
%%%%%%%%%%%%%%%%%%%%%%%%%%%%%%%%%%%%%%%%%%

\newcommand{\vc}[1]{\boldsymbol{#1}}
\newcommand{\adj}[1]{\frac{d J}{d #1}}
\newcommand{\chain}[2]{\adj{#2} = \adj{#1}\frac{d #1}{d #2}}

% mathcal
\newcommand{\Ac}{\mathcal{A}}
\newcommand{\Bc}{\mathcal{B}}
\newcommand{\Cc}{\mathcal{C}}
\newcommand{\Dc}{\mathcal{D}}
\newcommand{\Ec}{\mathcal{E}}
\newcommand{\Fc}{\mathcal{F}}
\newcommand{\Gc}{\mathcal{G}}
\newcommand{\Hc}{\mathcal{H}}
\newcommand{\Ic}{\mathcal{I}}
\newcommand{\Jc}{\mathcal{J}}
\newcommand{\Kc}{\mathcal{K}}
\newcommand{\Lc}{\mathcal{L}}
\newcommand{\Mc}{\mathcal{M}}
\newcommand{\Nc}{\mathcal{N}}
\newcommand{\Oc}{\mathcal{O}}
\newcommand{\Pc}{\mathcal{P}}
\newcommand{\Qc}{\mathcal{Q}}
\newcommand{\Rc}{\mathcal{R}}
\newcommand{\Sc}{\mathcal{S}}
\newcommand{\Tc}{\mathcal{T}}
\newcommand{\Uc}{\mathcal{U}}
\newcommand{\Vc}{\mathcal{V}}
\newcommand{\Wc}{\mathcal{W}}
\newcommand{\Xc}{\mathcal{X}}
\newcommand{\Yc}{\mathcal{Y}}
\newcommand{\Zc}{\mathcal{Z}}

% mathbb
\newcommand{\Ab}{\mathbb{A}}
\newcommand{\Bb}{\mathbb{B}}
\newcommand{\Cb}{\mathbb{C}}
\newcommand{\Db}{\mathbb{D}}
\newcommand{\Eb}{\mathbb{E}}
\newcommand{\Fb}{\mathbb{F}}
\newcommand{\Gb}{\mathbb{G}}
\newcommand{\Hb}{\mathbb{H}}
\newcommand{\Ib}{\mathbb{I}}
\newcommand{\Jb}{\mathbb{J}}
\newcommand{\Kb}{\mathbb{K}}
\newcommand{\Lb}{\mathbb{L}}
\newcommand{\Mb}{\mathbb{M}}
\newcommand{\Nb}{\mathbb{N}}
\newcommand{\Ob}{\mathbb{O}}
\newcommand{\Pb}{\mathbb{P}}
\newcommand{\Qb}{\mathbb{Q}}
\newcommand{\Rb}{\mathbb{R}}
\newcommand{\Sb}{\mathbb{S}}
\newcommand{\Tb}{\mathbb{T}}
\newcommand{\Ub}{\mathbb{U}}
\newcommand{\Vb}{\mathbb{V}}
\newcommand{\Wb}{\mathbb{W}}
\newcommand{\Xb}{\mathbb{X}}
\newcommand{\Yb}{\mathbb{Y}}
\newcommand{\Zb}{\mathbb{Z}}

% mathbf lowercase
\newcommand{\av}{\mathbf{a}}
\newcommand{\bv}{\mathbf{b}}
\newcommand{\cv}{\mathbf{c}}
\newcommand{\dv}{\mathbf{d}}
\newcommand{\ev}{\mathbf{e}}
\newcommand{\fv}{\mathbf{f}}
\newcommand{\gv}{\mathbf{g}}
\newcommand{\hv}{\mathbf{h}}
\newcommand{\iv}{\mathbf{i}}
\newcommand{\jv}{\mathbf{j}}
\newcommand{\kv}{\mathbf{k}}
\newcommand{\lv}{\mathbf{l}}
\newcommand{\mv}{\mathbf{m}}
\newcommand{\nv}{\mathbf{n}}
\newcommand{\ov}{\mathbf{o}}
\newcommand{\pv}{\mathbf{p}}
\newcommand{\qv}{\mathbf{q}}
\newcommand{\rv}{\mathbf{r}}
\newcommand{\sv}{\mathbf{s}}
\newcommand{\tv}{\mathbf{t}}
\newcommand{\uv}{\mathbf{u}}
\newcommand{\vv}{\mathbf{v}}
\newcommand{\wv}{\mathbf{w}}
\newcommand{\xv}{\mathbf{x}}
\newcommand{\yv}{\mathbf{y}}
\newcommand{\zv}{\mathbf{z}}

% mathbf uppercase
\newcommand{\Av}{\mathbf{A}}
\newcommand{\Bv}{\mathbf{B}}
\newcommand{\Cv}{\mathbf{C}}
\newcommand{\Dv}{\mathbf{D}}
\newcommand{\Ev}{\mathbf{E}}
\newcommand{\Fv}{\mathbf{F}}
\newcommand{\Gv}{\mathbf{G}}
\newcommand{\Hv}{\mathbf{H}}
\newcommand{\Iv}{\mathbf{I}}
\newcommand{\Jv}{\mathbf{J}}
\newcommand{\Kv}{\mathbf{K}}
\newcommand{\Lv}{\mathbf{L}}
\newcommand{\Mv}{\mathbf{M}}
\newcommand{\Nv}{\mathbf{N}}
\newcommand{\Ov}{\mathbf{O}}
\newcommand{\Pv}{\mathbf{P}}
\newcommand{\Qv}{\mathbf{Q}}
\newcommand{\Rv}{\mathbf{R}}
\newcommand{\Sv}{\mathbf{S}}
\newcommand{\Tv}{\mathbf{T}}
\newcommand{\Uv}{\mathbf{U}}
\newcommand{\Vv}{\mathbf{V}}
\newcommand{\Wv}{\mathbf{W}}
\newcommand{\Xv}{\mathbf{X}}
\newcommand{\Yv}{\mathbf{Y}}
\newcommand{\Zv}{\mathbf{Z}}

% bold greek lowercase
\newcommand{\alphav     }{\boldsymbol \alpha     }
\newcommand{\betav      }{\boldsymbol \beta      }
\newcommand{\gammav     }{\boldsymbol \gamma     }
\newcommand{\deltav     }{\boldsymbol \delta     }
\newcommand{\epsilonv   }{\boldsymbol \epsilon   }
\newcommand{\varepsilonv}{\boldsymbol \varepsilon}
\newcommand{\zetav      }{\boldsymbol \zeta      }
\newcommand{\etav       }{\boldsymbol \eta       }
\newcommand{\thetav     }{\boldsymbol \theta     }
\newcommand{\varthetav  }{\boldsymbol \vartheta  }
\newcommand{\iotav      }{\boldsymbol \iota      }
\newcommand{\kappav     }{\boldsymbol \kappa     }
\newcommand{\varkappav  }{\boldsymbol \varkappa  }
\newcommand{\lambdav    }{\boldsymbol \lambda    }
\newcommand{\muv        }{\boldsymbol \mu        }
\newcommand{\nuv        }{\boldsymbol \nu        }
\newcommand{\xiv        }{\boldsymbol \xi        }
\newcommand{\omicronv   }{\boldsymbol \omicron   }
\newcommand{\piv        }{\boldsymbol \pi        }
\newcommand{\varpiv     }{\boldsymbol \varpi     }
\newcommand{\rhov       }{\boldsymbol \rho       }
\newcommand{\varrhov    }{\boldsymbol \varrho    }
\newcommand{\sigmav     }{\boldsymbol \sigma     }
\newcommand{\varsigmav  }{\boldsymbol \varsigma  }
\newcommand{\tauv       }{\boldsymbol \tau       }
\newcommand{\upsilonv   }{\boldsymbol \upsilon   }
\newcommand{\phiv       }{\boldsymbol \phi       }
\newcommand{\varphiv    }{\boldsymbol \varphi    }
\newcommand{\chiv       }{\boldsymbol \chi       }
\newcommand{\psiv       }{\boldsymbol \psi       }
\newcommand{\omegav     }{\boldsymbol \omega     }

% bold greek uppercase
\newcommand{\Gammav     }{\boldsymbol \Gamma     }
\newcommand{\Deltav     }{\boldsymbol \Delta     }
\newcommand{\Thetav     }{\boldsymbol \Theta     }
\newcommand{\Lambdav    }{\boldsymbol \Lambda    }
\newcommand{\Xiv        }{\boldsymbol \Xi        }
\newcommand{\Piv        }{\boldsymbol \Pi        }
\newcommand{\Sigmav     }{\boldsymbol \Sigma     }
\newcommand{\Upsilonv   }{\boldsymbol \Upsilon   }
\newcommand{\Phiv       }{\boldsymbol \Phi       }
\newcommand{\Psiv       }{\boldsymbol \Psi       }
\newcommand{\Omegav     }{\boldsymbol \Omega     }

%%%%%%%%%%%%%%%%%%%%%%%%%%%%%%%%%%%%%%%%%%%
% Code highlighting with listings         %
%%%%%%%%%%%%%%%%%%%%%%%%%%%%%%%%%%%%%%%%%%%

\definecolor{bluekeywords}{rgb}{0.13,0.13,1}
\definecolor{greencomments}{rgb}{0,0.5,0}
\definecolor{redstrings}{rgb}{0.9,0,0}
\definecolor{light-gray}{gray}{0.95}

\newcommand{\MYhref}[3][blue]{\href{#2}{\color{#1}{#3}}}%

\definecolor{dkgreen}{rgb}{0,0.6,0}
\definecolor{gray}{rgb}{0.5,0.5,0.5}
\definecolor{mauve}{rgb}{0.58,0,0.82}

\lstdefinelanguage{Shell}{
  keywords={tar, cd, make},
  %keywordstyle=\color{bluekeywords}\bfseries,
  alsoletter={+},
  ndkeywords={python, py, javac, java, gcc, c, g++, cpp, .txt, octave, m, .tar},
  %ndkeywordstyle=\color{bluekeywords}\bfseries,
  identifierstyle=\color{black},
  sensitive=false,
  comment=[l]{//},
  morecomment=[s]{/*}{*/},
  commentstyle=\color{purple}\ttfamily,
  stringstyle=\color{red}\ttfamily,
  morestring=[b]',
  morestring=[b]",
  backgroundcolor = \color{light-gray}
}

\lstset{columns=fixed, basicstyle=\ttfamily,
    backgroundcolor=\color{light-gray},xleftmargin=0.5cm,frame=tlbr,framesep=4pt,framerule=0pt}



%%%%%%%%%%%%%%%%%%%%%%%%%%%%%%%%%%%%%%%%%%%
% Custom box for highlights               %
%%%%%%%%%%%%%%%%%%%%%%%%%%%%%%%%%%%%%%%%%%%

% Define box and box title style
\tikzstyle{mybox} = [fill=blue!10, very thick,
    rectangle, rounded corners, inner sep=1em, inner ysep=1em]

% \newcommand{\notebox}[1]{
% \begin{tikzpicture}
% \node [mybox] (box){%
%     \begin{minipage}{\textwidth}
%     #1
%     \end{minipage}
% };
% \end{tikzpicture}%
% }

\NewEnviron{notebox}{
\begin{tikzpicture}
\node [mybox] (box){
    \begin{minipage}{\textwidth}
        \BODY
    \end{minipage}
};
\end{tikzpicture}
}

%%%%%%%%%%%%%%%%%%%%%%%%%%%%%%%%%%%%%%%%%%%
% Commands showing / hiding solutions     %
%%%%%%%%%%%%%%%%%%%%%%%%%%%%%%%%%%%%%%%%%%%

%% To HIDE SOLUTIONS (to post at the website for students), set this value to 0: \def\issoln{0}
\def\issoln{0}
% Some commands to allow solutions to be embedded in the assignment file.
\ifcsname issoln\endcsname \else \def\issoln{0} \fi
% Default to an empty solutions environ.
\NewEnviron{soln}{}{}
% Default to an empty qauthor environ.
\NewEnviron{qauthor}{}{}
% Default to visible (but empty) solution box.
\newtcolorbox[]{studentsolution}[1][]{%
    breakable,
    enhanced,
    colback=white,
    title=Solution,
    #1
}

\if\issoln 1
% Otherwise, include solutions as below.
\RenewEnviron{soln}{
    \leavevmode\color{red}\ignorespaces
    \textbf{Solution} \BODY
}{}
\fi

\if\issoln 1
% Otherwise, include solutions as below.
\RenewEnviron{solution}{}
\fi

%%%%%%%%%%%%%%%%%%%%%%%%%%%%%%%%%%%%%%%%%%%
% Commands for customizing the assignment %
%%%%%%%%%%%%%%%%%%%%%%%%%%%%%%%%%%%%%%%%%%%

\newcommand{\courseNum}{\href{https://geometric3d.github.io}{16822}}
\newcommand{\courseName}{\href{https://geometric3d.github.io}{Geometry-based Methods in Vision}}
\newcommand{\courseSem}{\href{https://geometric3d.github.io}{Fall 2022}}
\newcommand{\courseUrl}{\url{https://piazza.com/cmu/fall2022/16822}}
\newcommand{\hwNum}{Problem Set 3}
\newcommand{\hwTopic}{Single View Geometry and Reconstruction}
\newcommand{\hwName}{\hwNum: \hwTopic}
\newcommand{\outDate}{Oct. 04, 2022}
\newcommand{\dueDate}{Oct. 11, 2022 11:59 PM}
\newcommand{\instructorName}{Shubham Tulsiani}
\newcommand{\taNames}{Mosam Dabhi, Kangle Deng, Jenny Nan}

%\pagestyle{fancyplain}
\lhead{\hwName}
\rhead{\courseNum}
\cfoot{\thepage{} of \numpages{}}

\title{\textsc{\hwName}} % Title


\author{}

\date{}

%%%%%%%%%%%%%%%%%%%%%%%%%%%%%%%%%%%%%%%%%%%%%%%%%
% Useful commands for typesetting the questions %
%%%%%%%%%%%%%%%%%%%%%%%%%%%%%%%%%%%%%%%%%%%%%%%%%

\newcommand \expect {\mathbb{E}}
\newcommand \mle [1]{{\hat #1}^{\rm MLE}}
\newcommand \map [1]{{\hat #1}^{\rm MAP}}
\newcommand \argmax {\operatorname*{argmax}}
\newcommand \argmin {\operatorname*{argmin}}
\newcommand \code [1]{{\tt #1}}
\newcommand \datacount [1]{\#\{#1\}}
\newcommand \ind [1]{\mathbb{I}\{#1\}}

\newcommand{\blackcircle}{\tikz\draw[black,fill=black] (0,0) circle (1ex);}
\renewcommand{\circle}{\tikz\draw[black] (0,0) circle (1ex);}

\newcommand{\pts}[1]{\textbf{[#1 pts]}}

%%%%%%%%%%%%%%%%%%%%%%%%%%
% Document configuration %
%%%%%%%%%%%%%%%%%%%%%%%%%%

% Don't display a date in the title and remove the white space
\predate{}
\postdate{}
\date{}

%%%%%%%%%%%%%%%%%%
% Begin Document %
%%%%%%%%%%%%%%%%%%


\begin{document}

\section*{}
\begin{center}
  \textsc{\LARGE \hwNum} \\
%   \textsc{\LARGE \hwTopic\footnote{Compiled on \today{} at \currenttime{}}} \\
  \vspace{1em}
  \textsc{\large \courseNum{} \courseName{} (\courseSem)} \\
  %\vspace{0.25em}
  \courseUrl\\
  \vspace{1em}
  OUT: \outDate \\
  DUE: \dueDate \\
  Instructor: \instructorName \\
  TAs: \taNames
\end{center}

\section*{START HERE: Instructions}
\begin{itemize}
\item \textbf{Collaboration policy:} All are encouraged to work together BUT you must do your own work (code and write up). If you work with someone, please include their name in your write up and cite any code that has been discussed. If we find highly identical write-ups or code without proper accreditation of collaborators, we will take action according to university policies, i.e. you will likely fail the course. See the \href{https://www.dropbox.com/s/z6o0tinc9eaez46/L01_Overview.pdf?dl=0}{Academic Integrity Section} detailed in the initial lecture for more information.

\item\textbf{Late Submission Policy:} There are \textbf{no} late days for Problem Set submissions.

\item\textbf{Submitting your work:}

\begin{itemize}

\item We will be using Gradescope (\url{https://gradescope.com/}) to submit the Problem Sets. Please use the provided template. Submissions can be written in LaTeX. Regrade requests can be made, however this gives the TA the opportunity to regrade your entire paper, meaning if additional mistakes are found then points will be deducted.
Each derivation/proof should be  completed on a separate page. For short answer questions you \textbf{should} include your work in your solution.  
\end{itemize}

\item \textbf{Materials:} The data that you will need in order to complete this assignment is posted along with the writeup and template on Piazza.

\end{itemize}

For multiple choice or select all that apply questions, replace \lstinline{\choice} with \lstinline{\CorrectChoice} to obtain a shaded box/circle, and don't change anything else.

\clearpage

\section*{Instructions for Specific Problem Types}

For ``Select One" questions, please fill in the appropriate bubble completely:

\begin{quote}
\textbf{Select One:} Who taught this course?
     \begin{checkboxes}
     \CorrectChoice Shubham Tulsiani
     \choice Deepak Pathak
     \choice Fernando De la Torre
     \choice Deva Ramanan
    \end{checkboxes}
\end{quote}

For ``Select all that apply" questions, please fill in all appropriate squares completely:

\begin{quote}
\textbf{Select all that apply:} Which are scientists?
{
    \checkboxchar{$\Box$} \checkedchar{$\blacksquare$}
    \begin{checkboxes}
     \CorrectChoice Stephen Hawking
     \CorrectChoice Albert Einstein
     \CorrectChoice Isaac Newton
     \choice None of the above
    \end{checkboxes}
    }
\end{quote}

For questions where you must fill in a blank, please make sure your final answer is fully included in the given space. You may cross out answers or parts of answers, but the final answer must still be within the given space.

\begin{quote}
\textbf{Fill in the blank:} What is the course number?

\begin{tcolorbox}[fit,height=1cm, width=4cm, blank, borderline={1pt}{-2pt},nobeforeafter, halign=center, valign=center]
    \begin{center}\huge16-822\end{center}
    \end{tcolorbox}\hspace{2cm}
\end{quote}

\clearpage

\section{Camera Models  [6 pts]}
\begin{questions}

\question \textbf{[4 pts]} Are these statements true or false?

(a) Any camera projection matrix $\Pv$ can be decomposed as $\Kv\Rv[\Iv | -\tilde{\Cv}]$.

(b) A camera with centre at infinity is an affine camera.

(c) If a $3\times 4$ matrix $\Pv$ represents an affine camera, then $rank(\Pv)=2$.

(d) $\Pv^{+}\uv$ corresponds to a point at infinity given a image location $\uv$ and a camera matrix $\Pv$, where $\Pv^{+} = \Pv^{\mathsf{T}}(\Pv\Pv^{\mathsf{T}})^{-1}$.

\begin{tcolorbox}[fit,height=2cm, width=\textwidth, blank, borderline={0.5pt}{-2pt},halign=left, valign=center, nobeforeafter]

(a) False\\
(b) False\\
(c) False\\
(d) False

\end{tcolorbox}

\question \textbf{[2 pts]} What is the direction of surface normal to the principal plane? 

(a) when expressed in terms of the elements of $\Pv = \begin{bmatrix} 
p_{11} & p_{12} & p_{13} & p_{14} \\ 
p_{21} & p_{22} & p_{23} & p_{24} \\ 
p_{31} & p_{32} & p_{33} & p_{34}
\end{bmatrix}$

(b) when expressed in terms of the elements of the intrinsic matrix $\Kv$, rotation $\Rv$, and camera centre $\tilde{\Cv}$.

For both the above, your answer need not be a normalized vector.

\begin{tcolorbox}[fit,height=2cm, width=\textwidth, blank, borderline={0.5pt}{-2pt},halign=left, valign=center, nobeforeafter]
(a) Surface normal = \begin{bmatrix} p_{31}&p_{32}&p_{33} \end{bmatrix}^T\\
(b) Surface normal = $m^3$, where $m^3$ is the third row vector of the matrix M = KR
\end{tcolorbox}

\section{Single-view Geometry [6 pts]}
\question \textbf{[4 pts]} How many 3D to 2D correspondences would be required to determine the camera matrix $\Pv$ in each of these cases? Which of these cases can be solved by reducing the constraints on the unknown variables to equations of the form $\Av\xv=\bv$, or $\Av\xv=\mathbf{0}~\text{s.t.}~||x||=1$?

(a) $\Pv$ matrix given all unknowns.

(b) Known camera rotation and intrinsics, but unknown camera translation.

(c) Known camera extrinsics, and unknown camera intrinsics but with the information that pixels are square.

(d) $\Pv$ matrix given $\Pv$ corresponds to an affine camera.

\begin{tcolorbox}[fit,height=3cm, width=\textwidth, blank, borderline={0.5pt}{-2pt},halign=left, valign=center, nobeforeafter]

(a) 6; can be solved by reducing to $\Av\xv=\mathbf{0}~\text{s.t.}~||x||=1$\\
(b) 2; can be solved by reducing to $\Av\xv=\mathbf{0}~\text{s.t.}~||x||=1$\\
(c) 2; can be solved by reducing to $\Av\xv=\mathbf{0}~\text{s.t.}~||x||=1$\\
(d) 4; can be solved by reducing to $\Av\xv=\bv$\\
\end{tcolorbox}

\question \textbf{[2 pts]} Given a line $\lv$ in an image, what is the equation of the plane that projects to $\lv$? You may assume $\Pv$ is known.

\begin{tcolorbox}[fit,height=5cm, width=\textwidth, blank, borderline={0.5pt}{-2pt},halign=left, valign=center, nobeforeafter]

Assuming point $\xv_1$ lies on the line $\lv$ in the image 
\begin{equation}\label{eq:line-point}
    \Rightarrow \lv^T \xv_1 = 0
\end{equation}
3-D projection of point, $\Xv_1 = \Pv^{-1} \xv_1$
The plane this point exists on can be given as
\begin{equation}\label{plane-point}
    \piv^T \Xv_1 = 0 \Rightarrow \hspace{0.2cm} \piv^T\Pv^{-1} \xv_1 = 0  
\end{equation}

Comparing equations \ref{eq:line-point} and \ref{plane-point}, we can say $\lv^T = \piv^T\Pv^{-1}\Rightarrow$ \textcolor{red}{$\piv = \Pv^T\lv$}

\end{tcolorbox}

\section{Single-view Reconstruction [8 pts]}

\question \textbf{[2 pts]} Show that the absolute conic in $\mathbb{P}^3$ is unchanged under a 3D euclidean transform.

\begin{tcolorbox}[fit,height=5cm, width=\textwidth, blank, borderline={0.5pt}{-2pt},halign=left, valign=center, nobeforeafter]

The absolute conic in the projective space is given as  $\Omegav\equiv \begin{bmatrix} \dv& 0 \end{bmatrix}^T;\hspace{0.2cm} \dv^T \mathbb{I}_3 \dv=0$\\
If we assume a 3D euclidean transformation, in the projective space, it would be, $\Tv = \begin{bmatrix} \Rv \hspace{0.2cm} |& \tv \end{bmatrix}$, where $\Rv$ is a $3\times3$ rotation matrix and $\tv$ is a $3\times1$ translation matrix.\\
Transformed absolute conic, $\Omegav^{'} \equiv \Tv \Omegav \Rightarrow\hspace{0.2cm} \Omegav^{'} \equiv \begin{bmatrix} \Rv_{3\times3} \hspace{0.2cm} |& \tv_{3\times1} \end{bmatrix}\begin{bmatrix} \dv_{3\times1}& 0_{1\times1} \end{bmatrix}^T \Rightarrow \hspace{0.2cm} \Omegav^{'}\equiv \Rv \dv$\\
Checking the condition, $(\Rv \dv)^T \mathbb{I}_3  (\Rv \dv) =0\Rightarrow \hspace{0.2cm} \dv^T \Rv^T \mathbb{I}_3 \Rv \dv =0$

\begin{equation}\label{eq:conic_rot}
    \Rightarrow\hspace{0.2cm} \dv^T \Rv^T \Rv \dv =0
\end{equation}

As rotation matrices are orthogonal matrices $\Rightarrow \Rv^T\Rv = \mathbb{I}_3 $\\
This means, the equation \ref{eq:conic_rot} will be converted to $\dv^T \mathbb{I}_3 \dv=0$. This has returned the original absolute conic equation.\\
\textcolor{red}{$\Rightarrow$ The absolute conic remains unchanged under a euclidean transform}

\end{tcolorbox}

\question \textbf{[4 pts]} Are these statements true or false?

(a) The image of absolute conic $\omega$ only comprises of imaginary points in the image plane.

(b) The rays corresponding to pixels $\uv_1$ and $\uv_2$ are orthogonal if and only if $\uv_1^T \omega \uv_2 = 0$.

(c) Assuming zero skew, the IAC can be identified given 4 sets of vanishing points for line pairs in orthogonal directions.

(d) The homography between the image plane and the plane at infinity does not depend on the camera location.

\begin{tcolorbox}[fit,height=5cm, width=\textwidth, blank, borderline={0.5pt}{-2pt},halign=left, valign=center, nobeforeafter]

(a) True\\
(b) True\\
(c) True\\
(d) True

\end{tcolorbox}

\question \textbf{[2 pts]} Given $\Kv=diag(2,2,1)$, find the normal direction of a plane whose vanishing line is given by $[1,2,3]^T$.

\begin{tcolorbox}[fit,height=5cm, width=\textwidth, blank, borderline={0.5pt}{-2pt},halign=left, valign=center, nobeforeafter]
The relation between the intrinsics $\Kv$, vanishing line $\lv$ for a plane and its normal $\nv$ is given as $\lv = \Kv^{-T} \nv$, in euclidean space.\\

$\Rightarrow \nv = \Kv^T \lv = \begin{bmatrix} 2&0&0\\0&2&0\\0&0&1 \end{bmatrix}^T \begin{bmatrix} 1\\2\\3 \end{bmatrix} =  \begin{bmatrix} 2&0&0\\0&2&0\\0&0&1 \end{bmatrix} \begin{bmatrix} 1\\2\\3 \end{bmatrix}$\\

\textcolor{red}{$\Rightarrow \nv = \begin{bmatrix} 2&4&3 \end{bmatrix}^T$}

\end{tcolorbox}

\end{questions}

\clearpage

\textbf{Collaboration Questions} Please answer the following:

\begin{enumerate}
    \item Did you receive any help whatsoever from anyone in solving this assignment?
    \begin{checkboxes}
     \choice Yes
     \CorrectChoice No
    \end{checkboxes}
    \begin{itemize}
        \item If you answered `Yes', give full details:
        \item (e.g. “Jane Doe explained to me what is asked in Question 3.4”)
    \end{itemize}

    \begin{tcolorbox}[fit,height=3cm,blank, borderline={1pt}{-2pt},nobeforeafter]
    %Input your solution here.  Do not change any of the specifications of this solution box.
    \end{tcolorbox}

    \item Did you give any help whatsoever to anyone in solving this assignment?
    \begin{checkboxes}
     \choice Yes
     \CorrectChoice No
    \end{checkboxes}
    \begin{itemize}
        \item If you answered `Yes', give full details:
        \item (e.g. “I pointed Joe Smith to section 2.3 since he didn’t know how to proceed with Question 2”)
    \end{itemize}

    \begin{tcolorbox}[fit,height=3cm,blank, borderline={1pt}{-2pt},nobeforeafter]
    %Input your solution here.  Do not change any of the specifications of this solution box.
    \end{tcolorbox}

    \item Did you find or come across code that implements any part of this assignment ? 
    \begin{checkboxes}
     \choice Yes
     \CorrectChoice No
    \end{checkboxes}
    \begin{itemize}
        \item If you answered `Yes', give full details: \underline{No}
        \item (book \& page, URL \& location within the page, etc.).
    \end{itemize}
    \begin{tcolorbox}[fit,height=3cm,blank, borderline={1pt}{-2pt},nobeforeafter]
    %Input your solution here.  Do not change any of the specifications of this solution box.
    \end{tcolorbox}
\end{enumerate}

\end{document}