\documentclass[11pt,addpoints,answers]{exam}
\usepackage[margin=1in]{geometry}
\usepackage{amsmath, amsfonts}
\usepackage{enumerate}
\usepackage{graphicx}
\usepackage{titling}
\usepackage{url}
\usepackage{xfrac}
\usepackage{geometry}
\usepackage{graphicx}
\usepackage{natbib}
\usepackage{amsmath}
\usepackage{amssymb}
\usepackage{amsthm}
\usepackage{paralist}
\usepackage{epstopdf}
\usepackage{tabularx}
\usepackage{longtable}
\usepackage{multirow}
\usepackage{multicol}
\usepackage[colorlinks=true,urlcolor=blue]{hyperref}
\usepackage{fancyvrb}
\usepackage{algorithm}
\usepackage{algorithmic}
\usepackage{float}
\usepackage{paralist}
\usepackage[svgname]{xcolor}
\usepackage{enumerate}
\usepackage{array}
\usepackage{times}
\usepackage{url}
\usepackage{comment}
\usepackage{environ}
\usepackage{times}
\usepackage{textcomp}
\usepackage{caption}
\usepackage[colorlinks=true,urlcolor=blue]{hyperref}
\usepackage{listings}
\usepackage{parskip} % For NIPS style paragraphs.
\usepackage[compact]{titlesec} % Less whitespace around titles
\usepackage[inline]{enumitem} % For inline enumerate* and itemize*
\usepackage{datetime}
\usepackage{comment}
% \usepackage{minted}
\usepackage{lastpage}
\usepackage{color}
\usepackage{xcolor}
\usepackage{listings}
\usepackage{tikz}
\usetikzlibrary{shapes,decorations,bayesnet}
%\usepackage{framed}
\usepackage{booktabs}
\usepackage{cprotect}
\usepackage{xcolor}
\usepackage{verbatimbox}
\usepackage[many]{tcolorbox}
\usepackage{cancel}
\usepackage{wasysym}
\usepackage{mdframed}
\usepackage{subcaption}
\usetikzlibrary{shapes.geometric}

%%%%%%%%%%%%%%%%%%%%%%%%%%%%%%%%%%%%%%%%%%%
% Formatting for \CorrectChoice of "exam" %
%%%%%%%%%%%%%%%%%%%%%%%%%%%%%%%%%%%%%%%%%%%

\CorrectChoiceEmphasis{}
\checkedchar{\blackcircle}

%%%%%%%%%%%%%%%%%%%%%%%%%%%%%%%%%%%%%%%%%%%
% Better numbering                        %
%%%%%%%%%%%%%%%%%%%%%%%%%%%%%%%%%%%%%%%%%%%

\numberwithin{equation}{section} % Number equations within sections (i.e. 1.1, 1.2, 2.1, 2.2 instead of 1, 2, 3, 4)
\numberwithin{figure}{section} % Number figures within sections (i.e. 1.1, 1.2, 2.1, 2.2 instead of 1, 2, 3, 4)
\numberwithin{table}{section} % Number tables within sections (i.e. 1.1, 1.2, 2.1, 2.2 instead of 1, 2, 3, 4)


%%%%%%%%%%%%%%%%%%%%%%%%%%%%%%%%%%%%%%%%%%%
% Common Math Commands                    %
%%%%%%%%%%%%%%%%%%%%%%%%%%%%%%%%%%%%%%%%%%%

%%%%%%%%%%%%%%%%%%%%%%%%%%%%%%%%%%%%%%%%%%
% Custom commands                        %
%%%%%%%%%%%%%%%%%%%%%%%%%%%%%%%%%%%%%%%%%%

\newcommand{\vc}[1]{\boldsymbol{#1}}
\newcommand{\adj}[1]{\frac{d J}{d #1}}
\newcommand{\chain}[2]{\adj{#2} = \adj{#1}\frac{d #1}{d #2}}

% mathcal
\newcommand{\Ac}{\mathcal{A}}
\newcommand{\Bc}{\mathcal{B}}
\newcommand{\Cc}{\mathcal{C}}
\newcommand{\Dc}{\mathcal{D}}
\newcommand{\Ec}{\mathcal{E}}
\newcommand{\Fc}{\mathcal{F}}
\newcommand{\Gc}{\mathcal{G}}
\newcommand{\Hc}{\mathcal{H}}
\newcommand{\Ic}{\mathcal{I}}
\newcommand{\Jc}{\mathcal{J}}
\newcommand{\Kc}{\mathcal{K}}
\newcommand{\Lc}{\mathcal{L}}
\newcommand{\Mc}{\mathcal{M}}
\newcommand{\Nc}{\mathcal{N}}
\newcommand{\Oc}{\mathcal{O}}
\newcommand{\Pc}{\mathcal{P}}
\newcommand{\Qc}{\mathcal{Q}}
\newcommand{\Rc}{\mathcal{R}}
\newcommand{\Sc}{\mathcal{S}}
\newcommand{\Tc}{\mathcal{T}}
\newcommand{\Uc}{\mathcal{U}}
\newcommand{\Vc}{\mathcal{V}}
\newcommand{\Wc}{\mathcal{W}}
\newcommand{\Xc}{\mathcal{X}}
\newcommand{\Yc}{\mathcal{Y}}
\newcommand{\Zc}{\mathcal{Z}}

% mathbb
\newcommand{\Ab}{\mathbb{A}}
\newcommand{\Bb}{\mathbb{B}}
\newcommand{\Cb}{\mathbb{C}}
\newcommand{\Db}{\mathbb{D}}
\newcommand{\Eb}{\mathbb{E}}
\newcommand{\Fb}{\mathbb{F}}
\newcommand{\Gb}{\mathbb{G}}
\newcommand{\Hb}{\mathbb{H}}
\newcommand{\Ib}{\mathbb{I}}
\newcommand{\Jb}{\mathbb{J}}
\newcommand{\Kb}{\mathbb{K}}
\newcommand{\Lb}{\mathbb{L}}
\newcommand{\Mb}{\mathbb{M}}
\newcommand{\Nb}{\mathbb{N}}
\newcommand{\Ob}{\mathbb{O}}
\newcommand{\Pb}{\mathbb{P}}
\newcommand{\Qb}{\mathbb{Q}}
\newcommand{\Rb}{\mathbb{R}}
\newcommand{\Sb}{\mathbb{S}}
\newcommand{\Tb}{\mathbb{T}}
\newcommand{\Ub}{\mathbb{U}}
\newcommand{\Vb}{\mathbb{V}}
\newcommand{\Wb}{\mathbb{W}}
\newcommand{\Xb}{\mathbb{X}}
\newcommand{\Yb}{\mathbb{Y}}
\newcommand{\Zb}{\mathbb{Z}}

% mathbf lowercase
\newcommand{\av}{\mathbf{a}}
\newcommand{\bv}{\mathbf{b}}
\newcommand{\cv}{\mathbf{c}}
\newcommand{\dv}{\mathbf{d}}
\newcommand{\ev}{\mathbf{e}}
\newcommand{\fv}{\mathbf{f}}
\newcommand{\gv}{\mathbf{g}}
\newcommand{\hv}{\mathbf{h}}
\newcommand{\iv}{\mathbf{i}}
\newcommand{\jv}{\mathbf{j}}
\newcommand{\kv}{\mathbf{k}}
\newcommand{\lv}{\mathbf{l}}
\newcommand{\mv}{\mathbf{m}}
\newcommand{\nv}{\mathbf{n}}
\newcommand{\ov}{\mathbf{o}}
\newcommand{\pv}{\mathbf{p}}
\newcommand{\qv}{\mathbf{q}}
\newcommand{\rv}{\mathbf{r}}
\newcommand{\sv}{\mathbf{s}}
\newcommand{\tv}{\mathbf{t}}
\newcommand{\uv}{\mathbf{u}}
\newcommand{\vv}{\mathbf{v}}
\newcommand{\wv}{\mathbf{w}}
\newcommand{\xv}{\mathbf{x}}
\newcommand{\yv}{\mathbf{y}}
\newcommand{\zv}{\mathbf{z}}

% mathbf uppercase
\newcommand{\Av}{\mathbf{A}}
\newcommand{\Bv}{\mathbf{B}}
\newcommand{\Cv}{\mathbf{C}}
\newcommand{\Dv}{\mathbf{D}}
\newcommand{\Ev}{\mathbf{E}}
\newcommand{\Fv}{\mathbf{F}}
\newcommand{\Gv}{\mathbf{G}}
\newcommand{\Hv}{\mathbf{H}}
\newcommand{\Iv}{\mathbf{I}}
\newcommand{\Jv}{\mathbf{J}}
\newcommand{\Kv}{\mathbf{K}}
\newcommand{\Lv}{\mathbf{L}}
\newcommand{\Mv}{\mathbf{M}}
\newcommand{\Nv}{\mathbf{N}}
\newcommand{\Ov}{\mathbf{O}}
\newcommand{\Pv}{\mathbf{P}}
\newcommand{\Qv}{\mathbf{Q}}
\newcommand{\Rv}{\mathbf{R}}
\newcommand{\Sv}{\mathbf{S}}
\newcommand{\Tv}{\mathbf{T}}
\newcommand{\Uv}{\mathbf{U}}
\newcommand{\Vv}{\mathbf{V}}
\newcommand{\Wv}{\mathbf{W}}
\newcommand{\Xv}{\mathbf{X}}
\newcommand{\Yv}{\mathbf{Y}}
\newcommand{\Zv}{\mathbf{Z}}

% bold greek lowercase
\newcommand{\alphav     }{\boldsymbol \alpha     }
\newcommand{\betav      }{\boldsymbol \beta      }
\newcommand{\gammav     }{\boldsymbol \gamma     }
\newcommand{\deltav     }{\boldsymbol \delta     }
\newcommand{\epsilonv   }{\boldsymbol \epsilon   }
\newcommand{\varepsilonv}{\boldsymbol \varepsilon}
\newcommand{\zetav      }{\boldsymbol \zeta      }
\newcommand{\etav       }{\boldsymbol \eta       }
\newcommand{\thetav     }{\boldsymbol \theta     }
\newcommand{\varthetav  }{\boldsymbol \vartheta  }
\newcommand{\iotav      }{\boldsymbol \iota      }
\newcommand{\kappav     }{\boldsymbol \kappa     }
\newcommand{\varkappav  }{\boldsymbol \varkappa  }
\newcommand{\lambdav    }{\boldsymbol \lambda    }
\newcommand{\muv        }{\boldsymbol \mu        }
\newcommand{\nuv        }{\boldsymbol \nu        }
\newcommand{\xiv        }{\boldsymbol \xi        }
\newcommand{\omicronv   }{\boldsymbol \omicron   }
\newcommand{\piv        }{\boldsymbol \pi        }
\newcommand{\varpiv     }{\boldsymbol \varpi     }
\newcommand{\rhov       }{\boldsymbol \rho       }
\newcommand{\varrhov    }{\boldsymbol \varrho    }
\newcommand{\sigmav     }{\boldsymbol \sigma     }
\newcommand{\varsigmav  }{\boldsymbol \varsigma  }
\newcommand{\tauv       }{\boldsymbol \tau       }
\newcommand{\upsilonv   }{\boldsymbol \upsilon   }
\newcommand{\phiv       }{\boldsymbol \phi       }
\newcommand{\varphiv    }{\boldsymbol \varphi    }
\newcommand{\chiv       }{\boldsymbol \chi       }
\newcommand{\psiv       }{\boldsymbol \psi       }
\newcommand{\omegav     }{\boldsymbol \omega     }

% bold greek uppercase
\newcommand{\Gammav     }{\boldsymbol \Gamma     }
\newcommand{\Deltav     }{\boldsymbol \Delta     }
\newcommand{\Thetav     }{\boldsymbol \Theta     }
\newcommand{\Lambdav    }{\boldsymbol \Lambda    }
\newcommand{\Xiv        }{\boldsymbol \Xi        }
\newcommand{\Piv        }{\boldsymbol \Pi        }
\newcommand{\Sigmav     }{\boldsymbol \Sigma     }
\newcommand{\Upsilonv   }{\boldsymbol \Upsilon   }
\newcommand{\Phiv       }{\boldsymbol \Phi       }
\newcommand{\Psiv       }{\boldsymbol \Psi       }
\newcommand{\Omegav     }{\boldsymbol \Omega     }

%%%%%%%%%%%%%%%%%%%%%%%%%%%%%%%%%%%%%%%%%%%
% Code highlighting with listings         %
%%%%%%%%%%%%%%%%%%%%%%%%%%%%%%%%%%%%%%%%%%%

\definecolor{bluekeywords}{rgb}{0.13,0.13,1}
\definecolor{greencomments}{rgb}{0,0.5,0}
\definecolor{redstrings}{rgb}{0.9,0,0}
\definecolor{light-gray}{gray}{0.95}

\newcommand{\MYhref}[3][blue]{\href{#2}{\color{#1}{#3}}}%

\definecolor{dkgreen}{rgb}{0,0.6,0}
\definecolor{gray}{rgb}{0.5,0.5,0.5}
\definecolor{mauve}{rgb}{0.58,0,0.82}

\lstdefinelanguage{Shell}{
  keywords={tar, cd, make},
  %keywordstyle=\color{bluekeywords}\bfseries,
  alsoletter={+},
  ndkeywords={python, py, javac, java, gcc, c, g++, cpp, .txt, octave, m, .tar},
  %ndkeywordstyle=\color{bluekeywords}\bfseries,
  identifierstyle=\color{black},
  sensitive=false,
  comment=[l]{//},
  morecomment=[s]{/*}{*/},
  commentstyle=\color{purple}\ttfamily,
  stringstyle=\color{red}\ttfamily,
  morestring=[b]',
  morestring=[b]",
  backgroundcolor = \color{light-gray}
}

\lstset{columns=fixed, basicstyle=\ttfamily,
    backgroundcolor=\color{light-gray},xleftmargin=0.5cm,frame=tlbr,framesep=4pt,framerule=0pt}



%%%%%%%%%%%%%%%%%%%%%%%%%%%%%%%%%%%%%%%%%%%
% Custom box for highlights               %
%%%%%%%%%%%%%%%%%%%%%%%%%%%%%%%%%%%%%%%%%%%

% Define box and box title style
\tikzstyle{mybox} = [fill=blue!10, very thick,
    rectangle, rounded corners, inner sep=1em, inner ysep=1em]

% \newcommand{\notebox}[1]{
% \begin{tikzpicture}
% \node [mybox] (box){%
%     \begin{minipage}{\textwidth}
%     #1
%     \end{minipage}
% };
% \end{tikzpicture}%
% }

\NewEnviron{notebox}{
\begin{tikzpicture}
\node [mybox] (box){
    \begin{minipage}{\textwidth}
        \BODY
    \end{minipage}
};
\end{tikzpicture}
}

%%%%%%%%%%%%%%%%%%%%%%%%%%%%%%%%%%%%%%%%%%%
% Commands showing / hiding solutions     %
%%%%%%%%%%%%%%%%%%%%%%%%%%%%%%%%%%%%%%%%%%%

%% To HIDE SOLUTIONS (to post at the website for students), set this value to 0: \def\issoln{0}
\def\issoln{0}
% Some commands to allow solutions to be embedded in the assignment file.
\ifcsname issoln\endcsname \else \def\issoln{0} \fi
% Default to an empty solutions environ.
\NewEnviron{soln}{}{}
% Default to an empty qauthor environ.
\NewEnviron{qauthor}{}{}
% Default to visible (but empty) solution box.
\newtcolorbox[]{studentsolution}[1][]{%
    breakable,
    enhanced,
    colback=white,
    title=Solution,
    #1
}

\if\issoln 1
% Otherwise, include solutions as below.
\RenewEnviron{soln}{
    \leavevmode\color{red}\ignorespaces
    \textbf{Solution} \BODY
}{}
\fi

\if\issoln 1
% Otherwise, include solutions as below.
\RenewEnviron{solution}{}
\fi

%%%%%%%%%%%%%%%%%%%%%%%%%%%%%%%%%%%%%%%%%%%
% Commands for customizing the assignment %
%%%%%%%%%%%%%%%%%%%%%%%%%%%%%%%%%%%%%%%%%%%

\newcommand{\courseNum}{\href{https://geometric3d.github.io}{16822}}
\newcommand{\courseName}{\href{https://geometric3d.github.io}{Geometry-based Methods in Vision}}
\newcommand{\courseSem}{\href{https://geometric3d.github.io}{Fall 2022}}
\newcommand{\courseUrl}{\url{https://piazza.com/cmu/fall2022/16822}}
\newcommand{\hwNum}{Problem Set 5}
\newcommand{\hwTopic}{N-view Geometry}
\newcommand{\hwName}{\hwNum: \hwTopic}
\newcommand{\outDate}{Nov. 8, 2022}
\newcommand{\dueDate}{Nov. 15, 2022 11:59 PM}
\newcommand{\instructorName}{Shubham Tulsiani}
\newcommand{\taNames}{Mosam Dabhi, Kangle Deng, Jenny Nan}

%\pagestyle{fancyplain}
\lhead{\hwName}
\rhead{\courseNum}
\cfoot{\thepage{} of \numpages{}}

\title{\textsc{\hwName}} % Title


\author{}

\date{}

%%%%%%%%%%%%%%%%%%%%%%%%%%%%%%%%%%%%%%%%%%%%%%%%%
% Useful commands for typesetting the questions %
%%%%%%%%%%%%%%%%%%%%%%%%%%%%%%%%%%%%%%%%%%%%%%%%%

\newcommand \expect {\mathbb{E}}
\newcommand \mle [1]{{\hat #1}^{\rm MLE}}
\newcommand \map [1]{{\hat #1}^{\rm MAP}}
\newcommand \argmax {\operatorname*{argmax}}
\newcommand \argmin {\operatorname*{argmin}}
\newcommand \code [1]{{\tt #1}}
\newcommand \datacount [1]{\#\{#1\}}
\newcommand \ind [1]{\mathbb{I}\{#1\}}

\newcommand{\blackcircle}{\tikz\draw[black,fill=black] (0,0) circle (1ex);}
\renewcommand{\circle}{\tikz\draw[black] (0,0) circle (1ex);}

\newcommand{\pts}[1]{\textbf{[#1 pts]}}

%%%%%%%%%%%%%%%%%%%%%%%%%%
% Document configuration %
%%%%%%%%%%%%%%%%%%%%%%%%%%

% Don't display a date in the title and remove the white space
\predate{}
\postdate{}
\date{}

%%%%%%%%%%%%%%%%%%
% Begin Document %
%%%%%%%%%%%%%%%%%%


\begin{document}

\section*{}
\begin{center}
  \textsc{\LARGE \hwNum} \\
%   \textsc{\LARGE \hwTopic\footnote{Compiled on \today{} at \currenttime{}}} \\
  \vspace{1em}
  \textsc{\large \courseNum{} \courseName{} (\courseSem)} \\
  %\vspace{0.25em}
  \courseUrl\\
  \vspace{1em}
  OUT: \outDate \\
  DUE: \dueDate \\
  Instructor: \instructorName \\
  TAs: \taNames
\end{center}

\section*{START HERE: Instructions}
\begin{itemize}
\item \textbf{Collaboration policy:} All are encouraged to work together BUT you must do your own work (code and write up). If you work with someone, please include their name in your write up and cite any code that has been discussed. If we find highly identical write-ups or code without proper accreditation of collaborators, we will take action according to university policies, i.e. you will likely fail the course. See the \href{https://www.dropbox.com/s/z6o0tinc9eaez46/L01_Overview.pdf?dl=0}{Academic Integrity Section} detailed in the initial lecture for more information.

\item\textbf{Late Submission Policy:} There are \textbf{no} late days for Problem Set submissions.

\item\textbf{Submitting your work:}

\begin{itemize}

\item We will be using Gradescope (\url{https://gradescope.com/}) to submit the Problem Sets. Please use the provided template. Submissions can be written in LaTeX. Regrade requests can be made, however this gives the TA the opportunity to regrade your entire paper, meaning if additional mistakes are found then points will be deducted.
Each derivation/proof should be  completed on a separate page. For short answer questions you \textbf{should} include your work in your solution.  
\end{itemize}

\item \textbf{Materials:} The data that you will need in order to complete this assignment is posted along with the writeup and template on Piazza.
s
\end{itemize}
\clearpage

\section*{Instructions for Specific Problem Types}

For ``Select One" questions, please fill in the appropriate bubble completely:

\begin{quote}
\textbf{Select One:} Who taught this course?
     \begin{checkboxes}
     \CorrectChoice Shubham Tulsiani
     \choice Deepak Pathak
     \choice Fernando De la Torre
     \choice Deva Ramanan
    \end{checkboxes}
\end{quote}

For ``Select all that apply" questions, please fill in all appropriate squares completely:

\begin{quote}
\textbf{Select all that apply:} Which are scientists?
{
    \checkboxchar{$\Box$} \checkedchar{$\blacksquare$}
    \begin{checkboxes}
     \CorrectChoice Stephen Hawking
     \CorrectChoice Albert Einstein
     \CorrectChoice Isaac Newton
     \choice None of the above
    \end{checkboxes}
    }
\end{quote}

For questions where you must fill in a blank, please make sure your final answer is fully included in the given space. You may cross out answers or parts of answers, but the final answer must still be within the given space.

\begin{quote}
\textbf{Fill in the blank:} What is the course number?

\begin{tcolorbox}[fit,height=1cm, width=4cm, blank, borderline={1pt}{-2pt},nobeforeafter, halign=center, valign=center]
    \begin{center}\huge16-822\end{center}
    \end{tcolorbox}\hspace{2cm}
\end{quote}

\clearpage

\section{Auto-calibration [10 pts]}
\begin{questions}

\question \textbf{[8 pts]} How many images are required to compute a metric rectification in each of the following scenarios (please give brief 1-2 line explanations):

(a) The principal point in all images is known to be at $(0,0)$.

(b) The cameras have zero skew.

(c) The principal point in all images is known to be at $(0,0)$, and the cameras have zero skew.

(d)  Camera intrinsics are known for all cameras.

(e) Camera instrinsics for each camera are of the form
$
\begin{bmatrix}
f & 0 & 0 \\
0 & f & 0 \\
0 & 0 & 0
\end{bmatrix}$, but each camera can have a different and unknown focal length.

(f) The intrinsics across all cameras are known to be the same.

(g) (2pts) The intrinsics across all cameras are known to be the same and the plane at infinity is known in the 3D reconstruction.

\begin{tcolorbox}[fit,height=5cm, width=\textwidth, blank, borderline={0.5pt}{-2pt},halign=left, valign=center, nobeforeafter]

(a) 5; 2 principal points give 2 constraints, we need 9 for $\Lv$\\
(b) 9; zero skew gives a single constraint\\
(c) 3; the above two combined give 3 constrints in total\\
(d) 2; known camera intrinsics give us 5 constraints from a single image\\
(e) 3; this form means zero skew, principal point = $(0,0)$ and equal focal lengths for each camera, this yields 4 constraints in each image. This could also be done with 2 images and a polynomial solution over a family of $\Lv$ with the constraint $det(\Lv)=0$, similar to the 7-point algorithm\\
(f) 3; a set of images gives 5 non-linear constraints, 2 sets of images can give all the 9 required constraints\\
(g) 2; We need 2 set of images to resolve 5 degrees of freedom and then the plane at infinity gives three additional degrees of freedoms
\end{tcolorbox}


\question \textbf{[2 pts]} To recover a metric rectification, we need to compute a Homography $H$ of the form $\begin{bmatrix}
K_1 & \mathbf{0} \\
\mathbf{v}^{\mathsf{T}} & 0 \\
\end{bmatrix}$. Defining $A=\begin{bmatrix}
K_1 \\
\mathbf{v}^{\mathsf{T}} \\
\end{bmatrix}$, our strategy is to first determine $L=A^{\mathsf{T}}A$ using constraints on intrinsic matrices, and then recover A. Given $L$, outline the approach for computing $A$.

\begin{tcolorbox}[fit,height=4.5cm, width=\textwidth, blank, borderline={0.5pt}{-2pt},halign=left, valign=center, nobeforeafter]

Given $A=\begin{bmatrix}K_1 \\ \mathbf{v}^{\mathsf{T}} \end{bmatrix}$ , $\Lv = \Av \Av^T = \begin{bmatrix} K_1 K_1^T & K_1 \vv \\ (K_1 \vv)^T & \vv^T \vv 
\end{bmatrix}$\\ 
\vspace{0.2cm}
\begin{itemize}
    \item $\dv \dv^T = cholesky(\Lv_{3\times3 sub})$ ; $K_1 = \dv^T$
    \item $K_1 \vv = \Lv_{3 \times 1 \text{ top right}} \Rightarrow$ $\vv = K_1^{-1} \Lv_{3 \times 1 \text{ top right}}$
\end{itemize}

As we have got $K_1$ and $\vv$ $A$ can be achieved by stacking them up as $A=\begin{bmatrix} K_1 \\ \mathbf{v}^{\mathsf{T}} \end{bmatrix}$

\end{tcolorbox}

\newpage
\section{Non-linear Least Squares [4 pts]}

\question \textbf{[4 pts]} Are these statements true or false?  

(a) Gauss-Newton method is a second-order optimization method.

(b) If we optimize $\min_\xv \|A\xv-\bv\|^2$ using Gauss-Newton, we will find the global optima in a single iteration. 

(c) Gauss-Newton optimization can be used for minimizing any cost function $\min_{\xv} C(\xv)$.

(d) When performing bundle adjustment, we have an objective that is typically of the form of multiple re-projection errors measuring the difference between the observed and re-projected 2D coordinates: $L = \sum \| \xv^{obs}_{n} - \xv^{proj}_n \|$, where this optimization is iteratively performed via a algorithm like Gauss-Newton. Is the following statement true or false -- if we  add a (positive) weight associated with each error i.e. $L = w_n \sum \| \xv^{obs}_{n} - \xv^{proj}_n \|$, then we can no longer use Gauss-Newton to optimize this objective.

\begin{tcolorbox}[fit,height=2cm, width=\textwidth, blank, borderline={0.5pt}{-2pt},halign=left, valign=center, nobeforeafter]

(a) False\\
(b) True\\
(c) False\\
(d) False\\

\end{tcolorbox}

\section{Trifocal Tensor [5 pts]}

\question \textbf{[2 pts]} Are these statements true or false?  

(a) The trifocal tensfor corresponding to the triplet of cameras $(P, P', P'')$ is the same as the trifocal tensfor for $(PH, P'H, P''H)$, where $H$ is any invertible matrix.

(b) Assuming $\Tv = [\Tv_1, \Tv_2, \Tv_3]$ is the trifocal tensor for $(P, P', P'')$, then $ \Tv_1^{\mathsf{T}}, \Tv_2^{\mathsf{T}}, \Tv_3^{\mathsf{T}}$ is the trifocal tensor for $(P, P'', P')$. 

\begin{tcolorbox}[fit,height=2cm, width=\textwidth, blank, borderline={0.5pt}{-2pt},halign=left, valign=center, nobeforeafter]

(a) True\\
(b) True

\end{tcolorbox}

\question \textbf{[3 pts]} Let us assume that $\Tv = [\Tv_1, \Tv_2, \Tv_3]$ is the trifocal tensor for $(P, P', P'')$. We then transform the camera $P''$ to $\bar{P}''$ by rotating it in place by $R$ (i.e. camera center and intrinsics $K''$ are unchanged). Find the trifocal tensfor for the triplet $(P, P', \bar{P}'')$ in terms of $ [\Tv_1, \Tv_2, \Tv_3], K, R$.

\begin{tcolorbox}[fit,height=5cm, width=\textwidth, blank, borderline={0.5pt}{-2pt},halign=left, valign=center, nobeforeafter]
Given the camera $P''$ is rotated to $\bar{P}''$ in place, this transformation can be given\\
by a homography as $\bar{\xv}'' = \Hv  \xv''$ , $\Hv = KRK^{-1}$\\

Given the trifocal tensor for $(P, P', P'')$ , we know , 
\begin{equation} \label{eq::trifoc}
\lv = \lv'^T [\Tv_1, \Tv_2, \Tv_3] \lv''
\end{equation}
For $\bar{P}''$ , $\bar{\lv}'' = \Hv ^{-T} \lv'' \Rightarrow$ $\lv'' = \Hv^T \bar{\lv}'' = K^{-T} R^T K^T \bar{\lv}''$\\

Substituting this in equation \ref{eq::trifoc}, we get the trifocal tensor for the triplet $(P, P', \bar{P}'')$ as\\

$\lv = \lv'^T [\Tv_1, \Tv_2, \Tv_3] K^{-T} R^T K^T \bar{\lv}''$\\

New trifocal tensor is \textcolor{blue}{$[\Tv_1, \Tv_2, \Tv_3] K^{-T} R^T K^T$}

\end{tcolorbox}
\newpage
\section{Structure from Motion [6 pts]}

\question \textbf{[3 pts]} Are these statements true or false?  

(a) Given sufficient images of a static scene with accurate correspondences, one can recover a metric reconstruction of the underlying scene.

(b) When performing Hierarchical SfM, we first obtain independent reconstructions across clusters and then align them. If we know camera intrinsics for all cameras, two independently reconstructed clusters would need to be aligned via a euclidean transform.

(c) When performing incremental SfM, the initial two cameras selected for bootstrapping should ideally correspond to the closest possible pair.

\begin{tcolorbox}[fit,height=2cm, width=\textwidth, blank, borderline={0.5pt}{-2pt},halign=left, valign=center, nobeforeafter]

(a) False\\
(b) True\\
(c) False

\end{tcolorbox}


\question \textbf{[3 pts]} Affine Triangulation: Assuming known affine cameras $\{\Pv_n\}$ and observed projections $\{\xv_n\}$ for an (unknown) 3D point $\Xv$, describe an algorithm to compute $\Xv$ that minimizes the sum of squared reprojection error across all cameras.

\begin{tcolorbox}[fit,height=7cm, width=\textwidth, blank, borderline={0.5pt}{-2pt},halign=left, valign=center, nobeforeafter]
The algorithm to minimize reprojection error in affine triangulation is 
\begin{itemize}
    \item Compute translation $\tv = \langle \xv \rangle = \frac{1}{n} \Sigma \xv_n $
    \item Center the projections, each $\xv_n \leftarrow \xv_n - \tv$
    \item Construct the $\Wv$ matrix of size $2 \mv \times 1$, where $\mv$ is the number of cameras; $\Wv = \begin{bmatrix} \xv^1_1 \\ \xv^2_1 \\ \vdots \\ \xv^1_n \\\xv^2_n \end{bmatrix}$
    \item As the cameras are known we have the camera matrix stack $\Mv = \begin{bmatrix} P_1\\ \vdots \\ P_n \end{bmatrix}$ of size $2\mv \times 3$
    \item This makes up the equation $\Wv = \Mv \Xv$. As we have to triangulate a single point, this can be formulated as a least squares $\min ||\Mv \Xv - \Wv||^2 = 0$
    \item Solution is $\Xv = (\Mv^T \Mv)^{-1} \Mv^T \Wv$
\end{itemize}

\end{tcolorbox}


\end{questions}

\clearpage

\textbf{Collaboration Questions} Please answer the following:

\begin{enumerate}
    \item Did you receive any help whatsoever from anyone in solving this assignment?
    \begin{checkboxes}
     \CorrectChoice Yes
     \choice No
    \end{checkboxes}
    \begin{itemize}
        \item If you answered `Yes', give full details:
        \item (e.g. “Jane Doe explained to me what is asked in Question 3.4”)
    \end{itemize}
    
    \begin{tcolorbox}[fit,height=3cm,blank, borderline={1pt}{-2pt},nobeforeafter]
    %Input your solution here.  Do not change any of the specifications of this solution box.
    I recieved help in Q5 concept from Siddhartha Namburu (snamburu)
    \end{tcolorbox}

    \item Did you give any help whatsoever to anyone in solving this assignment?
    \begin{checkboxes}
     \choice Yes
     \CorrectChoice No
    \end{checkboxes}
    \begin{itemize}
        \item If you answered `Yes', give full details:
        \item (e.g. “I pointed Joe Smith to section 2.3 since he didn’t know how to proceed with Question 2”)
    \end{itemize}

    \begin{tcolorbox}[fit,height=3cm,blank, borderline={1pt}{-2pt},nobeforeafter]
    %Input your solution here.  Do not change any of the specifications of this solution box.
    \end{tcolorbox}

    \item Did you find or come across code that implements any part of this assignment ? 
    \begin{checkboxes}
     \choice Yes
     \CorrectChoice No
    \end{checkboxes}
    \begin{itemize}
        \item If you answered `Yes', give full details: \underline{No}
        \item (book \& page, URL \& location within the page, etc.).
    \end{itemize}
    \begin{tcolorbox}[fit,height=3cm,blank, borderline={1pt}{-2pt},nobeforeafter]
    %Input your solution here.  Do not change any of the specifications of this solution box.
    \end{tcolorbox}
\end{enumerate}

\end{document}